\documentclass{writeup}

\begin{document}
\tagged{greedy}

\begin{solutions}
  \begin{solution}{The ultimate greedy solution}{n!}{n}
    The ultimate naive solution is to try every possible configuration of boats, check whether it's valid and keep the one with the most boats assigned to their ring.
    This is not trivial to code as it would require some recursion to construct the cross product of sets of possible boat positions.
    This would require $O(n!)$ time and is therefore not feasable for large $n$.
  \end{solution}
  
  \begin{solution}{A naive greedy solution}{n^2\log{n}}{1}
    The naive solution keeps track of the next boat that will be the best to assign to its ring.
    The best boat to assign next is the one that fits the tightest to the next free point.

    \code{naive}{6}{32}
  \end{solution}

  \begin{solution}{A better solution}{n}{1}
    Because the above greedy solution is correct, eventhough it is too slow, we can use it in a proof.
    To build a better solution, first we have to realise that if we want to place a new boat, and it would require replacing two previous boats, that boat can never appear in an optimal solution.

    This means that we can keep track of the rightmost points of the previous two boats that were placed.
    \code{main}{19}{25}
    If the next boat we check doesnt' collide with any of the previous two boats, we can just place the boat.
    \code{main}{26}{29}
    If it collides with the last boat and would take up less space, we replace it.
    \code{main}{30}{38}
    In both of these situations the new boat has to be in the optimal solution.
  \end{solution}
\end{solutions}



\end{document}
