\documentclass[guide.tex]{subfiles}
\begin{document}

\section{Mathematics}

\begin{matht}{Even - Odd}
  In a binary number system, a number is even if and only if its last digit is zero.
  Similarly: a number is odd if and only if its last digit is one.

  \example{even_odd}
\end{matht}

\begin{matht}{Sum of integers}
  \[
    \sum_{k = 1}^{n}k = \frac{n(n+1)}{2}
    \quad
    \text{ and }
    \quad
    \sum_{k = 0}^{n - 1}k = \frac{n^2}{2}
  \]
\end{matht}

\begin{matht}{Sum of squares}
  \[
    \sum_{k = 1}^{n}k^2 = \frac{n(n+1)(2n+1)}{6}
    \quad
    \text{ and }
    \quad
    \sum_{k = 0}^{n - 1}k^2 = \frac{n(2n^2+2n+1)}{6}
  \]
\end{matht}

\begin{matht}{Sum of cubes}
  \[
    \sum_{k = 1}^{n}k^3 = \left(\sum_{k = 1}^{n}k\right)^2
  \]
\end{matht}

\begin{matht}{Recursive exponentiation}
  \[
    a^n
    =
    \begin{cases}
      1 & n = 0\\
      a & n = 1\\
      \left(a^{\frac{n}{2}}\right)^{2} & n \text{ even}\\
      aa^{n-1} & n \text{ odd}\\
    \end{cases}
  \]
  Recursive implementation:
  \example{recursive_exponentiation}
  Iterative implementation:
  \example{iterative_exponentiation}

  Runtime: $O(\log(n))$, assuming multiplication is a constant-time operation.
\end{matht}

\begin{matht}{Greatest Common Divisor}
  Facts about the greatest common divisor:
  \begin{align*} 
    \forall a,b \in \mathbb{N}:\ & b \ge a \Rightarrow gcd(a, b) = gcd(a, b - a)\\
    \forall a \in \mathbb{N}:\ & gcd(a, 0) = a
  \end{align*}

  Recursive implementation:
  \example{recursive_gcd}
  Iterative implementation:
  \example{iterative_gcd}
  Using boost:
  \example{boost_gcd}

  Runtime: $O(\log(a + b))$
\end{matht}

\begin{matht}{Least Common Multiple}
  The least common multiple can be expressed in terms of the greatest common denominator.
  \[ lcm(a, b) = \frac{ab}{gcd(a, b)} \]

  \example{lcm_ito_gcd}
  Using boost:
  \example{boost_lcm}

  Runtime: $O(\log(a + b))$
\end{matht}

\end{document}

%%% Local Variables:
%%% mode: latex
%%% TeX-master: t
%%% End:
