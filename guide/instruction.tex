\documentclass[guide.tex]{subfiles}
\begin{document}


\section{Configuration}

\subsection{Vim}

\begin{minted}{raw}
  syntax on

  " size of a hard tabstop
  set tabstop=2

  " size of an "indent"
  set shiftwidth=2

  " a combination of spaces and tabs are used to simulate tab stops at a width
  " other than the (hard)tabstop
  set softtabstop=2

  " make "tab" insert indents instead of tabs at the beginning of a line
  set smarttab

  " always uses spaces instead of tab characters
  set expandtab

\end{minted}


\section{Solving a problem}

{\Huge

\begin{enumerate}
  \item Read the assignment until the formal specification.
  \item Write the IO while reading the formal specification.
  \item Come up with a model for the solution using the following pages.
\end{enumerate}

}

\begin{itemize}
\item Put this at every problem.
  \newline \cc{std::ios_base::sync_with_stdio(false);}
\item Compile with one of these commands:
  \newline \cl{cgal_create_cmake_script; cmake .; make}
  \begin{minted}{raw}
g++ -o main.bin main.cc
  -Wall -Werror
  -std=c++11
  -lm
  -lboost_system -lboost_graph -lboost_thread
  -lCGAL -lgmp -lmpfr -frounding-math -lCGAL_Core
  \end{minted}
\end{itemize}
\pagebreak

\section{Models}
\subsection{Greedy}

\begin{itemize}
  \item Make sure any local optimum is a global optimum.
  \item Define the locally optimal choice.
\end{itemize}

\subsection{Divide and conquer}
\begin{itemize}
  \item Define how to divide.
  \item Define how to solve the simplest base-case.
  \item Define how to merge the results.
\end{itemize}

\subsection{Dynamic programming}
\begin{itemize}
  \item Optional: Define a recursive solution with no optimization
  \item Define the state to optimize.
\end{itemize}

\subsection{Scan}
\begin{itemize}
  \item Define the events
	\item Make sure you can go over them without ever considering the same event twice.
\end{itemize}

\subsection{Binary search}
\begin{itemize}
  \item Make sure that the collection is sorted.
  \item Make sure that you have random access in $O(1)$ time to the collection's elements
\end{itemize}

\subsection{BFS}
\begin{itemize}
  \item Define nodes.
  \item Define edges.
  \item Define what 'visiting a node' means.
\end{itemize}

Use a queue to hold vertices that still need to be visited.
\begin{minted}{c++}
  #include <queue>
\end{minted}
Use a color to represent visitness: \cc{WHITE} represents unvisited, \cc{GRAY} represens 'on the queue' and \cc{BLACK} represents 'visited'.
If written like this, vertices will be initialized to \cc{WHITE} automatically.
\begin{minted}{c++}
  enum color { WHITE, GRAY, BLACK };
\end{minted}
This uses your own vertex representation, fill in \cc{V} with whatever you need.
Edges are not necessarily explicitly represented
\begin{minted}{c++}
  vector<color> visited(nr_vertices, WHITE);

  V start;
  // Initialize start

  queue<V> q;
  q.push(start);
  while (! q.empty()) {
    V& cur = q.front();
    q.pop()

    // Visit q
    for (/* all vertices that can be reached: */ V n) {
      if (visited[n] != WHITE) { continue; }
      q.push(n);
      visited[n] = GRAY;
    }
    visited[cur] = BLACK;
  }
\end{minted}

\subsection{DFS}
\begin{itemize}
  \item Define nodes.
  \item Define edges.
  \item Define what 'visiting a node' means.
\end{itemize}

Like BFS (see above) but with a stack:
\begin{minted}{c++}
  #include <stack>
  stack<V> s; // Initialize
  s.push(start);

  V& cur = s.top() // Get top element
  s.pop() // Remove top element

  s.push(n) // Add new elements to top
\end{minted}

\subsection{Linear/Quadratic programming}
\begin{minted}{c++}
  #include <CGAL/basic.h>
  #include <CGAL/QP_models.h>
  #include <CGAL/QP_functions.h>

  using namespace CGAL;

  #ifdef CGAL_USE_GMP
  #include <CGAL/Gmpz.h>
  typedef Gmpz ET;
  #else
  #include <CGAL/MP_Float.h>
  typedef MP_Float ET;
  #endif
\end{minted}
\begin{itemize}
  \item Decide whether it's an integer program or a real-valued program:
  \begin{itemize}
    \item Integer (most of the time)
\begin{minted}{c++}
  typedef Quadratic_program<int> Program;
  typedef Quadratic_program_solution<ET> Solution;
\end{minted}
    \item Real-valued
\begin{minted}{c++}
  typedef Quadratic_program<ET>                       Program;
  typedef Quadratic_program_solution<ET>              Solution;
\end{minted}
  \end{itemize}
  \item Define the quadratic program
\begin{minted}{c++}
    // Always a mizimization
    Program qp(SMALLER // Default constraint
      , true // Whether or not to obey the lower bound
      , 0 // Lower bound for variables
      , false // Whether or not to obey the upper bound
      , 0); // Upper bound
\end{minted}
  \item Define the variables and their bounds.
\begin{minted}{c++}
  const int X = 0; 
  const int Y = 1;
  // Obey upper bound of 4 for variable y
  qp.set_u(Y, true, 4);
  // Obey lower bound of 1 for variable y
  qp.set_u(Y, true, 1);
\end{minted}
  \item Define the objective function to minimize/maximize.
  \begin{itemize}
    \item  Define the constant
\begin{minted}{c++}
  qp.set_c0(64);                                          // +64
\end{minted}
    \item  Define the linear factors
\begin{minted}{c++}
  qp.set_c(Y, -32);                                       // -32y
\end{minted}
    \item Define the quadratic factors
\begin{minted}{c++}
  // Watch out: specify double of quadratic factors
  qp.set_d(X, X, 2); qp.set_d (Y, Y, 8);                  // x^2 + 4 y^2
\end{minted}
  \end{itemize}
  \item Define the constraints. The second argument is the constraint number, make sure these are uniquely used.
\begin{minted}{c++}
  qp.set_a(X, 0,  1);
  qp.set_a(Y, 0, 1);
  p.set_r(eq, SMALLER);
  qp.set_b(0, 7);  //  x + y  <= 7

  qp.set_a(X, 1, -1);
  qp.set_a(Y, 1, 2);
  p.set_r(eq, LARGER);
  qp.set_b(1, 4);  // -x + 2y >= 4
\end{minted}
  \item Solve the problem
    \begin{itemize}
      \item Strictly linear program
\begin{minted}{c++}
      Solution s = CGAL::solve_linear_program(p, ET())
\end{minted}
      \item Quadratic program
\begin{minted}{c++}
      Solution s = CGAL::solve_quadratic_program(p, ET());
\end{minted}
    \end{itemize}
  \item Use the solution
\begin{minted}{c++}
  // This is costly, make sure to turn it off to test speed.
  s.solves_quadratic_program(p); // Tests whether the solution is valid
  s.is_optimal() // Tests whether the solution is in fact optimal
  s.is_unbounded() // Tests whether there exists a minimal solution
  // Gets the objective value using the helper methed for exact types
  floor_to_double(s.objective_value());
\end{minted}
\end{itemize}

\subsection{Minimum Spanning Tree}
\begin{itemize}
  \item Check that an undirected graph with edge weigths is a fitting model.

\begin{minted}{c++}
  #include <boost/graph/adjacency_list.hpp>
  using namespace boost;

  typedef adjacency_list<
    vecS,
    vecS,
    undirectedS,
    no_property,
    property<edge_weight_t, int>>                   Graph;
  typedef graph_traits<Graph>::vertex_descriptor    Vertex;
  typedef graph_traits<Graph>::edge_descriptor      Edge;
  typedef graph_traits<Graph>::edge_iterator        EdgeIt;
  typedef property_map<Graph, edge_weight_t>::type  WeightMap;
\end{minted}
  \item Define the graph.
\begin{minted}{c++}
  Graph g(t);
  WeightMap ws = get(edge_weight, g);
\end{minted}
  \item Define vertices.
\begin{minted}{c++}
  Vertex v = add_vertex(g);
\end{minted}
  \item Define edges.
\begin{minted}{c++}
  Edge e; bool s;
  tie(e, s) = add_edge(from, to, g);
\end{minted}
  \item Define edge costs.
\begin{minted}{c++}
  wm[ed] = edge.weight;
\end{minted}
  \item Compute the MST
  \begin{itemize}
    \item With Kruskall: (Use this one if you can.)
\begin{minted}{c++}
  #include <boost/graph/kruskal_min_spanning_tree.hpp>
  vector<Edge> mst;
  kruskal_minimum_spanning_tree(gi, back_inserter(mst));

  vector<Edge>::iterator ebeg, eend = mst.end();

  for (ebeg = mst.begin(); ebeg != eend; ++ebeg) {
    *ebeg // Current edge
    source(*ebeg, g) // Get source of the current edge
    target(*ebeg, g) // Get target of the current edge
  }
\end{minted}
    \item With Prim:
\begin{minted}{c++}
  #include <boost/graph/prim_minimum_spanning_tree.hpp>
  vector<Vertex> p(n); // Predecessor map
  prim_minimum_spanning_tree(g, &p[0]);

  for (int from = 0; from < n; ++from) {
    int to = p[from];
    Edge e; bool success;
    tie(e, success) = edge(from, to, g);

    if (! success) { continue; } // No parent, not on the tree.
  }
\end{minted}
  \end{itemize}
\end{itemize}

\subsection{Graph matching}
\begin{itemize}
  \item Define nodes.
  \item Define edges.
  \item A matching in a graph is a set of edges without common vertices.

\begin{minted}{c++}
#include <boost/graph/adjacency_list.hpp>
#include <boost/graph/max_cardinality_matching.hpp>

using namespace boost;

typedef adjacency_list<vecS, vecS, undirectedS> Graph;
typedef graph_traits<Graph>::edge_descriptor Edge;
typedef graph_traits<Graph>::vertex_descriptor Vertex;

vector<Vertex> mate(v);
checked_edmonds_maximum_cardinality_matching(g, &mate[0]);
int size = matching_size(g, &mate[0]);
\end{minted}
\end{itemize}


\subsection{Topological sort}
\begin{itemize}
  \item Directed graph.
  \item Define nodes.
  \item Define edges.
\begin{minted}{c++}
  typedef adjacency_list< vecS, vecS, directedS, color_property<> > Graph;
  typedef boost::graph_traits<Graph>::vertex_descriptor Vertex;

  Graph g(n);
  Edge e; bool s;
  tie(e, s) = add_edge(from, to, g);

  typedef std::vector<Vertex> container;
  container c;
  topological_sort(g, std::back_inserter(c));

  cout << "A topological ordering: ";
  for ( container::reverse_iterator ii=c.rbegin(); ii!=c.rend(); ++ii)
    cout << index(*ii) << " ";
  cout << endl;
\end{minted}
\end{itemize}


\subsection{Network flow: Max flow}
\begin{itemize}
  \item Define nodes
  \item Define edges.
  \item Define the source and the sink.
  \item Define flow capacity for each edge
\end{itemize}

\subsection{Network flow: Max flow/Min cost}
\begin{itemize}
  \item Define nodes
  \item Define edges.
  \item Define the source and the sink.
  \item Define flow capacity for each edge.
  \item Define flow cost for each edge.
\end{itemize}


\subsection{Delaunay/Voronoi}
\begin{itemize}
  \item Define points/sites.
  \item Define distances
    % What can I query quickly in voronoi/delaunay?
\end{itemize}

\subsection{Minimum enclosing shapes}
\begin{itemize}
  \item Define the points
\end{itemize}


\subsection{Shortest paths}
\begin{itemize}
  \item TODO
\end{itemize}

\section{These snippets may come in handy}

\subsection{Floor and Ceil: regular}
\begin{minted}{c++}
double floor_to_double(const K::FT& x) {
  double a = std::floor(CGAL::to_double(x));
  while (a > x) a -= 1;
  while (a+1 <= x) a += 1;
  return a;
}
\end{minted}

\begin{minted}{c++}
double ceil_to_double(const K::FT& x) {
  double a = std::ceil(CGAL::to_double(x));
  while (a < x) a += 1;
  while (a-1 >= x) a -= 1;
  return a;
}
\end{minted}

\subsection{Floor and Ceil: Sqrt}
\begin{minted}{c++}
double ceilsqrt(const K::FT &x) {
  const K::FT sqrtx = sqrt(x); // exact
  double a = floor(CGAL::to_double(sqrtx)); // not
  while (a > sqrtx) a -= 1; while (a < sqrtx) a += 1;
  return a;
}
\end{minted}

\begin{minted}{c++}
double floorsqrt(const K::FT& x) {
  const K::FT sqrtx = sqrt(x); // exact
  double a = std::floor(CGAL::to_double(sqrtx));
  while (a > sqrtx) a -= 1; while (a + 1 <= sqrtx) a += 1;
  return a;
}
\end{minted}


\begin{minted}{c++}

\end{minted}


\begin{minted}{c++}

\end{minted}

\end{document}

%%% Local Variables:
%%% mode: latex
%%% TeX-master: t
%%% End:
