\documentclass[guide.tex]{subfiles}
\begin{document}

\section{Troubleshooting}
{\large Common problems and mistakes}\\

\begin{mt}{Not initialising variables}

  While developing on your local machine, you probably don't use the optimization flags \cl{O}, \cl{O2} or \cl{Ofast}.
  When you don't use optimization flags, the compiler sometimes sets variables to a sensible default value.
  The judge's compiler will use an optimization flag and therefore not initialize variables for you.
  The following snippet will then work as intended locally but fail horribly on the judge.

  \example{uninitialized}

  \begin{fx}
    Initialize variables.\\
    If you don't, annotate the declaration of the variable with a comment explaining why you didn't.
    To catch these errors early, use the \cl{-Wuninitialized} flag.
  \end{fx}
\end{mt}

\begin{mt}{Accidentally reusing variables}
  \example{accidental_reuse}
  \begin{fx}
    Scope variables correctly.\\
    If we had written \cc{int i = 0;} instead of \cc{i = 0;}, the compiler would have complained that \cc{i} was already defined and we would have caught the error immediately.
  \end{fx}
\end{mt}

\begin{mt}{Mixing up variables in nested loops}
  \example{mixing_variables}
  \begin{fx}
    Thing about the meaning of the loop while writing the loop.
    Don't think about the content of the loop just yet.
    Annotate your loops with a comment explaining their meaning.
  \end{fx}
\end{mt}

\begin{mt}{A semicolon after a loop}
  \example{loop_semicolon}
  \begin{fx}
    Always put brackets around a loop.\\
    This wil ensurethat you check behind the loop for a semicolon.
  \end{fx}
\end{mt}


\end{document}

%%% Local Variables:
%%% mode: latex
%%% TeX-master: t
%%% End:
