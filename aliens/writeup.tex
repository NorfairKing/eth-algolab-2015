\documentclass{writeup}

\begin{document}

\begin{solutions}
  \begin{solution}{A naive solution}{n(n + m)}{nm}
    Even the naive solution to this problem requires a considerable amount of work.
    The naive solution is to just go over all possible quadruples and count all the even sums of the bits inbetween them.

    The problem is modeled as a graph problem.
    The vertices are either humans or aliens.

    \code{naive}{14}{18}

    Between an alien and a human, an edge exists if the source feels superior to the destination.
    For aliens, this is easily checked.
    The alien feels directly superior to all the humans in its interval.

    \code{naive}{9}{12}

    For humans this is not so easy.
    We will build a lookup table for the aliens that a human feels superior to.

    \code{naive}{23}{38}

    To count the number of aliens that feel superior to all others, we will go over all aliens and search to a depth of $3$ (\mintinline{c++}{ALIEN_MEMORY}) in the graph for all the entities it feels superior to and cross them off the list.

    \code{naive}{40}{97}

    We then go through the list and check whether all the entities have been found:

    \code{naive}{100}{115}
  \end{solution}

  \begin{solution}{A much more simple solution}{m}{1}
    It turns out that there is no need to actually traverse the graph of superiority.

    The first key realisation is that if an alien $x$ has a superiority interval $\interval{a}{b}$ and another alien $y$ has superiority interval $\interval{c}{d}$ that is a subset of $\interval{a}{b}$, then alien $b$ can never feel superior to alien $a$.

    The next realisation is that the superiority intervals of the aliens must unite to the entire set of humans.
    
    The final realisation is that if the superiority interval is not contained within the superiority interval of any other alien, then that alien can reach every other alien in at most two steps and therefore all humans in at most 3.

    Putting it all together, we'll sort the intervals by left ends and then by size.

    \code{main}{6}{21}

    \code{main}{23}{24}

    Then we'll check whether the intervals unite to the entire set of humans.

    \code{main}{25}{35}

    Lastly we count all the intervals that partially overlap.

    \code{main}{37}{48}
  \end{solution}
\end{solutions}

\end{document}
