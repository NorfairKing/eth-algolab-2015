\documentclass{writeup}

\begin{document}

\begin{solutions}
  The input for this problem is quite complicated.
  I like to seperate I/O from logic so I wrote some representations.

  \code{main}{5}{11}
  
  There is absolutely no use in making a distinction between `greater than' and `less than'.
  It will make the rest of the code simpler if we eliminate one of them by converting them to the other and switching the pans.
  I chose the `greater than`:

  \code{main}{13}{25}

  \begin{solution}{A solution (?)}{kn}{n}
    The first thing to notice is that the false coin is never going to appear in a weighing that resulted in `equal'.
    We can therefore eliminate all the coins that appear in an `equal' weighing.

    \code{main}{27}{40}

    The algorithm works by maintaining two sets of suspects.
    One set for the coins that are suspected to be lighter and another set for the coins that are suspected to be heavier.

    The key insight is that in any `less than' weighing, the false coin must be on the left if it is lighter and it must be on the right if it is heavier.
    This means that we can intersect the `lighter' suspects with the left pans of all `less than' weighings and intersect the `heavier' suspects with the right pans of the `less than'  weighings to narrow down the suspects.

    If there is exactly one suspect left in the end, that's the false coin.
    If there are multiple then we cannot give a difinitive answer.

    \code{main}{53}{88}

    The intersection of two vectors is already implemented in \mintinline{c++}{<algorithm>}.

    \code{main}{42}{51}

    Because both sorting and \mintinline{c++}{std::set_intersection} happen in linear time, this intersection takes time proportional to $n$.
  \end{solution}
\end{solutions}

\end{document}
