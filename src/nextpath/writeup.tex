\documentclass{writeup}

\begin{document}
\tagged{BFS}

\begin{solutions}
  \begin{solution}{Reality simulation}{PM}{PN}
    The second shortest distance is the distance that the walker would travel that arrives second.
    This solution walks over the graph and returns the distance that the second traveler travelled when he arrives at the destination.

    The solution is more general than just `the second shortest'.
    It will work for the $p$-th shortest path.

    \code{main}{6}{6}

    Edges are modeled as nothing more than edges but vertices keep track of the distance that the last traveler has travelled when they arrived at the vertex.
    The algorithms also, seperately, keeps track of how many times each vertex has been visited.

    \code{main}{8}{16}

    The algorithm works by traversing the graph with breath first search.
    It starts at the \mintinline{c++}{from} vertex with distance $0$.
    If the algorithm arrives at a vertex that has already been visited $p$ times, it will not go onto its neighbors any further.

    \code{main}{28}{62}

    To speed up the resolution of neighbors, neighbors are precomputed for all vertices.

    \code{main}{18}{26}

  \end{solution}
\end{solutions}



\end{document}
