\documentclass{writeup}

\begin{document}

\begin{solutions}
  \begin{solution}{The naive solution}{n^{3}}{1}
    The naive solution is to check whether the sum of bits between $i$ and $j$ is even between all pairs $i$ and $j$.
    Instead of summing the bits, we can XOR them with the sum and check whether the sum is $0$ in the end.
    This speeds up the solution very slightly but it's still too slow.
    \code{naive}{21}{36}
  \end{solution}

  \begin{solution}{A better solution}{n^2}{1}
    The sum of the bits between a pair $i$ and $j$ can be computed online.
    That is, while looping over the pairs.
    There is no need to have a third loop to compute the sum.
    This saves us an entire order of magnitude.
    \code{better}{21}{34}
  \end{solution}

  \begin{solution}{A dynamic programming solution}{n^2}{n}
    The sum of bits between $i$ and $j$ could be computed if we had all the sums until a given $k$.
    Denote $s[k]$ as the sum of the first $k$ bits.
    We then find that the sum of bits between $i$ and $j$ is equal to $s[j] - s[i]$.
    This is slightly faster though not asymptotically but it will prepare us for the next part.
    \code{precomputed}{25}{41}
  \end{solution}

  \begin{solution}{A better dynamic programming solution}{n}{n}
    To get an even faster solution, we have to go through all pairs $(i,j)$ without really going through all pairs individually.
    It turns out that there is a way to find out how many sums will be even given just a $j$.
    $s[j] - s[i]$ will be even if and only if either both $s[j]$ and $s[i]$ or they are both odd.
    This means that, given a $j$, if $s[j]$ is even then there will be $e$ even sums where $e$ is the number of even sums below $j$.
    If $j$ is odd then there will be $j-e$ (the others) even sums.
    We can precompute this $e$ for the entire array of bits.
    \code{dynamic}{37}{62}
  \end{solution}

  \begin{solution}{A sophisticated mathematical solution}{n}{1}
    As it turns out we can get an even better solution (one that uses less memory) by using some clever combinatorics.
    The first thing we need to see is that between two even sums, the sum of the bits is equal to $2$.
    That means that between every pair of even bits, the sum will be even again.
    The same holds for odd sums.
    The amount of possible pairs in a set of $n$ number is exactly $\frac{n(n-1)}{2}$.
    This means we only have to count the amount of even and the amount of odd sums and calculate this expression.
    \code{main}{20}{31}
  \end{solution}
\end{solutions}

\end{document}
