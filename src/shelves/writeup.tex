\documentclass{writeup}

\begin{document}

\begin{solutions}
  The problem, in its simplest form, is as follows:
  Find $a, b \in \mathbb{N}$ such that $am + bn \le l$ and minimise $l-am-bm$ with $b$ as high as possible.

  \begin{solution}{A naive solution}{\frac{l^2}{mn}}{1}
    The naive solution is to go over every possible combination of short and long shelves and keep the one with the least amount of error.
    \code{naive}{22}{37}
  \end{solution}

  \begin{solution}{A better solution}{\frac{l}{n}}{1}
    Given an $b$, it is easy to see that we can calculate the maximum $a$ we can have.
    This will give us the error $e$ as well.
    This solution just goes over all possible $b$'s.
    \code{better}{5}{20}
  \end{solution}

  \begin{solution}{The dual better solution}{\frac{l}{m}}{1}
    Notice that we can do the same thing we did in the better solution but with $a$'s instead of $b$'s.
    This solution will become important later.
    \code{dual_better}{5}{20}
  \end{solution}

  \begin{solution}{A very clever trick}{\sqrt{l}}{\sqrt{l}}
    The better solution is fine when $n$ is large but when $n$ is small it still takes time proportional to $l$.
    Suppose we know that $n$ is smaller than $\sqrt{l}$
    We can then do it faster.

    First look back at the dual better solution.
    The loop goes through $\frac{l}{m}$ iterations but it only needs to go through $n$ at most.
    \code{main}{24}{40}

    To show this, consider a configuration where $a$ is larger than $n$:
    \[ a = n + x \text{ with } x > 0 \]
    Because $n$ is smaller than $\sqrt{l}$, $x$ has to be smaller than $m-n$ (and thus smaller than $n$) for the configuration to be valid.

    We can rewrite the total as follows:
    \[
      a m - b n
      = (n + x) m - b * n
      = nm - xm - bn
      = x m + (b + m) n
    \]
    This means that the solution is equivalent to a solution where $x < n$ that we have therefore already checked.

    Now we can just use the best algorithm depending on the situation:

    \code{main}{42}{48}
  \end{solution}
\end{solutions}

\end{document}
