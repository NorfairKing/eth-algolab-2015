\documentclass{writeup}

\begin{document}

\begin{solutions}
  \tagged{max-flow}
  \begin{solution}{A naive solution}{?}{?}
    This is modeled as a max-flow problem.
    Any max-flow represents a min-cut. (pun intended)

    Me any my friend can each pick any figure to start.
    This means that we'll calculate the flow for any pair of vertices and keep the one with the maximum flow.
    This flow will represent the situation with the smallest value accross the cut.

    \code{naive}{70}{85}

    Then we'll collect all the figures before the cut.

    \code{naive}{87}{116}

  \end{solution}

  \begin{solution}{Slightly smarter cutting}{?}{?}
    The important insight is that we're not really interested in the entire flow network, only in the cut.
    Given any pair of vertices that are on either end of the cut we're looking for, we'll get the same cut.
    Because a cut is by definition a partition of the vertices into two sets, this means we don't have to look at every pair of vertices but only at the pairs with a fixed vertex (let's say $0$).
    
    \code{main}{78}{91}
  \end{solution}
\end{solutions}

\end{document}
