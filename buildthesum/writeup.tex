\documentclass{writeup}

\begin{document}

\begin{solutions}
  \begin{solution}{The naive solution}{n}{1}
    The naive solution is to read in all the numbers as floating point numbers, calculate the sum and output the result.
    The problem with this solution is that it won't work when one of the number is significantly larger or smaller than the others.
    This precision problem is problem that is inherent to the floating point representation of numbers.
    \code{naive}{32}{39}
  \end{solution}

  \begin{solution}{A clever solution}{n}{1}
    Any integer representation doesn't have this precision problem so we will try to solve this problem using only integers.
    Depending on the amount of decimals that are required, we will use a different integer representation.
    We'll represent the numbers as integers with respect to a base factor that is dependent on the number with the most decimals.
    We can then do additions without loss of precision and convert the result back to a floating point number to get the result.
    \code{main}{24}{50}
  \end{solution}
\end{solutions}

\end{document}
